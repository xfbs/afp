\documentclass[
  ngerman,
  paper=a4,
  10pt,
  headings=small,
  DIV=15,
]{scrartcl}

% use utf-8 encoding.
\usepackage[utf8]{inputenc}

% setup main font.
\usepackage{fontspec}
\setsansfont{Helvetica Neue}[
  BoldFont = {Helvetica Neue Bold},
]
\renewcommand{\familydefault}{\sfdefault}

% space to leave in between answers.
\renewcommand{\arraystretch}{1.2}

% question environment (uses a table).
\newenvironment{question}[2]{
  \noindent
  \setcounter{answer}{1}
  \begin{tabular}{@{}cp{6.5cm}@{}}
  \textbf{#1} & \textbf{#2}\\
}{
  \end{tabular}
  \bigskip
}

\newcounter{answer}
\newcommand{\answer}[1]{
  \textbf{\Alph{answer}}
  \stepcounter{answer}
  & #1\\}

% quotes setup
\usepackage{babel}
\usepackage{csquotes}
\MakeOuterQuote{"}

\title{Prüfungsfragen im Prüfungsteil "Technische Kenntnisse" bei Prüfungen zum Erwerb von Amateurfunkzeugnissen der Klasse E}
\date{September 2006}

\begin{document}
  % title page
  \maketitle
    
  % first page
  \newpage
  \vspace*{\fill}
  \noindent
  Dieser Fragen- und Antwortenkatalog basiert auf §~4 Abs.~1 Amateurfunkgesetz (AFuG) in Verbindung mit §~4 der durch Artikel~1 Ziffer~2 der Ersten Verordnung zur Änderung der Amateurfunkverordnung vom 25.~August 2006 (BGBl.~I~S.~2070) geänderten Verordnung zum Gesetz über den Amateurfunk (AFuV) vom 15. Februar 2005 (BGBl.~I~S.~242) in der Form, wie sie am 1.~Februar 2007 in Kraft tritt. Aus dem Katalog ersichtliche Einzelheiten werden erst ab dem 1.~Februar 2007 bei Amateurfunkprüfungen umgesetzt bzw. angewendet. Dazu erfolgt vor dem 1.~Februar 2007 eine entsprechende Veröffentlichung im Amtsblatt der Bundesnetzagentur.
  
Dieser Fragen- und Antwortenkatalog unterliegt den Bestimmungen des §~5 des Urheberrechtsgesetzes (UrhG).
Er kann jederzeit erweitert und aktualisiert werden. Neuauflagen werden im Amtsblatt der Bundesnetzagentur bekannt gegeben.
  \newpage
  \tableofcontents
  \twocolumn
  \section{Allgemeine mathematische Grundkenntnisse und Größen}
  \subsection{Allgemeine mathematische Grundkenntnisse}
  
  \begin{question}{TA101}{0,042A entspricht}
    \answer{$42 \cdot 10^{-3}$ A.}
    \answer{$42 \cdot 10^3$ A.}
    \answer{$42 \cdot 10^{-2}$ A.}
    \answer{$42 \cdot 10^{-1}$ A.}
  \end{question}
  
  \begin{question}{TA102}{0,00042 A entspricht}
    \answer{$420 \cdot 10^{-6}$ A.}
    \answer{$420 \cdot 10^{6}$ A.}
    \answer{$420 \cdot 10^{-}$ A.}
    \answer{$42 \cdot 10^{-6}$ A.}
  \end{question}
  
  \begin{question}{TA103}{100 mW entspricht}
    \answer{$10^{-1}$ W.}
    \answer{0,001 W.}
    \answer{0,01 W.}
    \answer{$10^{-2}$ W.}
  \end{question}
  
  \subsection{Größen und Einheiten}
  
  \begin{question}{TA201}{Welche Einheit wird für die elektrische Spannung verwendet?}
    \answer{Volt (V)}
    \answer{Ampere (A)}
    \answer{Ohm (V)}
    \answer{Amperestunden (Ah)}  
  \end{question}
  
  \begin{question}{TA202}{Welche Einheit wird für die elektrische Ladung verwendet?}
    \answer{Amperesekunde (As)}
    \answer{Kilowatt (kW)}
    \answer{Joule (J)}
    \answer{Ampere (A)}  
  \end{question}
  
  \begin{question}{TA203}{Welche Einheit wird für die elektrische Leistung verwendet?}
    \answer{Watt (W)}
    \answer{Kilowattstunden (kWh)}
    \answer{Joule (J)}
    \answer{Amperestunden (Ah)}
  \end{question}
  
  \section{Elektrizitäts-, Elektromagnetismus- und Funktheorie}
  
  \subsection{Leiter, Halbleiter und Isolator}
  \subsection{Strom- und Spannungsquellen}
  \subsection{Elektrisches Feld}
  \subsection{Magnetisches Feld}
  \subsection{Elektromagnetisches Feld}
  \subsection{Sinusförmige Signale}
  \subsection{Nichtsinusförmige Signale}
  \subsection{Modulierte Signale}
  \subsection{Ohmsches Gesetz, Leistung und Energie}
  
  \section{Elektrische und elektronische Bauteile}
  \subsection{Widerstand}
  \subsection{Kondensator}
  \subsection{Spule}
  \subsection{Übertrager und Transformatoren}
  \subsection{Diode}
  \subsection{Transistor}
  
  \section{Elektronische Schaltungen und deren Merkmale}
  \subsection{Serien- und Parallelschaltung von Widerständen, Spulen und Kondensatoren}
  \subsection{Schwingkreise und Filter}
  \subsection{Stromversorgung}
  \subsection{Verstärker}
  \subsection{Modulator / Demodulator}
  \subsection{Oszillator}
  
  \section{Analoge und digitale Modulationsverfahren}
  \subsection{Amplitudenmodulation AM, SSB}
  \subsection{Frequenzmodulation}
  \subsection{Text-, Daten- und Bildübertragung}
  
  \section{Funk-Empfänger}
  \subsection{Einfach- und Doppelsuperhet-Empfänger}
  \subsection{Blockschaltbilder}
  \subsection{Betrieb und Funktionsweise einzelner Stufen}
  \subsection{Empfängermerkmale}
  
  \section{Funksender}
  \subsection{Blockschaltbilder}
  \subsection{Betrieb und Funktionsweise einzelner Stufen}
  \subsection{Betrieb und Funktionsweise von HF-Leistungsverstärkern}
  \subsection{Betrieb und Funktionsweise von HF-Transceivern}
  \subsection{Unerwünschte Aussendungen}
  
  \section{Antennen und Übertragungsleitungen}
  \subsection{Antennen}
  \subsection{Antennenmerkmale}
  \subsection{Übertragungsleitungen}
  \subsection{Anpassung, Transformation und Symmetrierung}
  
  \section{Wellenausbreitung und Ionosphäre}
  \subsection{Ionosphäre}
  \subsection{Kurzwellenausbreitung}
  \subsection{Wellenausbreitung oberhalb 30 MHz}
  
  \section{Messungen und Messinstrumente}
  \subsection{Messinstrumente}
  \subsection{Durchführung von Messungen}
  
  \section{Störemissionen, Störfestigkeit, Schutzanforderungen, Ursachen, Abhilfe}
  \subsection{Störungen elektronischer Geräte}
  \subsection{Ursachen für Störungen}
  \subsection{Maßnahmen zur Störungsbeseitigung}
  
  \section{Elektromagnetische Verträglichkeit, Anwendung, Personen- und Sachschutz}
  \subsection{Störfestigkeit}
  \subsection{Schutz von Personen}
  \subsection{Sicherheit}
\end{document}
